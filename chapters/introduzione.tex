\chapter{Introduzione}
\label{cap:introduzione}

Questo capitolo tratta ....\\

\noindent Esempio di utilizzo di un termine nel glossario \\
\gls{api}. \\

\noindent Esempio di citazione in linea \\
\cite{site:agile-manifesto}. \\

\noindent Esempio di citazione nel pie' di pagina \\
citazione\footcite{womak:lean-thinking} \\

\section{L'azienda}

\subsection{Storia e Origini}
Winning Technologies, in sigla Wintech Spa, è un'azienda fondata nel 1987 dal \gls{CEO \& founder} Massimo Gallotta. Inizialmente concepita come \gls{System Integrator}, Wintech ha avuto la sua sede principale a Padova, con ulteriori filiali a Milano, Bassano del Grappa e Pordenone. 
Fin dalla sua nascita, Wintech opera nel settore ICT, fornendo consulenza e soluzioni tecnologiche personalizzate per ottimizzare i processi aziendali.

\subsection{Servizi Offerti}
Wintech si distingue per la sua offerta diversificata, che include:

\begin{itemize}
    \item Consulenza strategica
    \item Progettazione e implementazione di soluzioni tecnologiche ad hoc
    \item Commercializzazione di prodotti ICT delle principali aziende del settore
\end{itemize}

\subsection{Partnership Strategiche}
Wintech collabora con diverse aziende e organizzazioni per ampliare la propria offerta e garantire un supporto efficace ai clienti. Alcuni partner strategici includono:

\begin{itemize}
    \item \textbf{Sistemi Pordenone, Udine, Vicenza}: Sinergia nella distribuzione e supporto delle soluzioni di Sistemi Spa.
    \item \textbf{Quadrivium}: Azienda specializzata in sistemi smart per la produzione e logistica, contribuendo all’implementazione di soluzioni MES e WMS per l’industria 4.0.
    \item \textbf{Visione Learning}: Fornisce una piattaforma innovativa di e-learning, con un focus su servizi di formazione certificata.
    \item \textbf{Menocarta}: Network esperto in trasformazione digitale e dematerializzazione, focalizzato sulla fatturazione elettronica.
\end{itemize}

\subsection{Settori di Competenza}
I principali settori in cui Wintech si specializza includono:

\begin{itemize}
    \item Social Business Collaboration
    \item Business Intelligence
    \item Cash \& Credit Management
    \item Gestione Documentale
    \item Sistemi ERP
    \item Soluzioni per Commercialisti e Consulenti del Lavoro
    \item Progettazione di Infrastrutture di Rete
    \item Servizi di Full Outsourcing e System Integration
\end{itemize}

\subsection{Evoluzione e Innovazione}
Wintech ha saputo adattarsi e crescere in un contesto in continua evoluzione. Tra i momenti chiave della sua storia si possono annoverare:

\begin{itemize}
    \item \textbf{1987}: Inizio delle attività nel settore bancario.
    \item \textbf{1995}: Sperimentazione della piattaforma Dominio IBM e avvio di progetti legati all’automazione dei pagamenti.
    \item \textbf{1997}: Avvio di una collaborazione strategica con Sistemi.
    \item \textbf{2005}: Apertura della sede di Milano, rafforzando la partnership con IBM.
    \item \textbf{2011}: Espansione con l’apertura di una nuova sede a Bassano e sviluppo continuo di nuove collaborazioni.
\end{itemize}

\subsection{Riconoscimenti e Successi}
Grazie a una strategia solida e a un continuo impegno nell'innovazione, Wintech ha ottenuto numerosi riconoscimenti nel settore ICT, consolidando la propria reputazione come partner affidabile e competente.

\subsection{Partner e Collaborazioni}
Tra i partner tecnici di Wintech figurano nomi di spicco del settore come:

\begin{itemize}
    \item IBM
    \item Microsoft
    \item HP
    \item VMware
    \item Nutanix
    \item Veeam
    \item HPE Aruba Networking
    \item e molti altri.
\end{itemize}

Queste collaborazioni permettono a Wintech di fornire soluzioni di alta qualità e di mantenere una posizione competitiva nel mercato.

\subsection{Conclusioni}
Con oltre 35 anni di esperienza, Wintech continua a rappresentare un punto di riferimento nel panorama ICT, spingendo costantemente verso l’innovazione e la digitalizzazione. L'azienda è ben posizionata per affrontare le sfide future, offrendo ai propri clienti soluzioni all'avanguardia e un servizio di alta qualità.


 

\section{L'idea}

Introduzione all'idea dello stage.

\section{Organizzazione del testo}

\begin{description}
    \item[{\hyperref[cap:processi-metodologie]{Il secondo capitolo}}] descrive ...
    
    \item[{\hyperref[cap:descrizione-stage]{Il terzo capitolo}}] approfondisce ...
    
    \item[{\hyperref[cap:analisi-requisiti]{Il quarto capitolo}}] approfondisce ...
    
    \item[{\hyperref[cap:progettazione-codifica]{Il quinto capitolo}}] approfondisce ...
    
    \item[{\hyperref[cap:verifica-validazione]{Il sesto capitolo}}] approfondisce ...
    
    \item[{\hyperref[cap:conclusioni]{Nel settimo capitolo}}] descrive ...
\end{description}

Riguardo la stesura del testo, relativamente al documento sono state adottate le seguenti convenzioni tipografiche:
\begin{itemize}
	\item gli acronimi, le abbreviazioni e i termini ambigui o di uso non comune menzionati vengono definiti nel glossario, situato alla fine del presente documento;
	\item per la prima occorrenza dei termini riportati nel glossario viene utilizzata la seguente nomenclatura: \emph{parola}\glsfirstoccur;
	\item i termini in lingua straniera o facenti parti del gergo tecnico sono evidenziati con il carattere \emph{corsivo}.
\end{itemize}
